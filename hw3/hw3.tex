\documentclass[12pt]{article}
\usepackage[utf8]{inputenc, }
\usepackage{graphicx}
\usepackage{hyperref}
\usepackage[margin=1in]{geometry}
\usepackage{setspace}
\usepackage{color}
\usepackage{pdfpages}
\usepackage{amsmath}
\usepackage{amsfonts}
\usepackage{float}
\usepackage{tikz}
\usepackage{pgfplots}
\usepackage{enumitem}
\usepackage{xpatch}
\usepackage{svg}
\usepackage{mathrsfs}
% \usepackage[nocheck]{fancyhdr}

\sloppy
\definecolor{lightgray}{gray}{0.5}
\setlength{\parindent}{0pt}

\hypersetup{
    colorlinks,
    citecolor=black,
    filecolor=black,
    linkcolor=black,
    urlcolor=black
    pdftitle={EE300 İsmail Enes Bülbül}
}
\onehalfspacing

% \raggedright

\title{EE301 Homework-3}
\author{İsmail Enes Bülbül, Eren Meydanlı, Ahmet Caner Akar}
%\date{October 2022}
\renewcommand*\contentsname{Table of Contents}
\renewcommand*{\refname}{}
% \fancyhf{} % sets both header and footer to nothing
% \renewcommand{\headrulewidth}{0pt}
\begin{document}

\maketitle
% \tableofcontents
% \newpage

    \section*{Question 1}
    \subsection*{a)}
    \subsection*{b)}
    \section*{Question 2}
    \subsection*{a)}
    \subsection*{b)}
    \subsection*{c)}
    \section*{Question 3}
    \subsection*{a)}
    \subsubsection*{i)}
    \begin{math}
    x(t) = \frac{sin(4\pi t)}{\pi t} cos(2\pi t) = \frac{4sin(4\pi t)}{4\pi t} cos(2\pi t) = x_1(t)x_2(t) \left(\textrm{ where }x_1(t) = \frac{4sin(4\pi t)}{4\pi t} \textrm{, } x_2(t) = cos(2\pi t)\right)\\ \\ 
    \textrm{Recall that: } \mathscr{F}\{rect(\theta)\} = \frac{sin(\omega/2)}{\omega/2} \\
    \textrm{By duality property of CTFT: }\frac{sin(t/2)}{t/2} \longleftrightarrow 2\pi rect(-\omega) = 2\pi rect(\omega) \\
    \textrm{By scaling property of CTFT: }\frac{sin(4\pi t)}{4\pi t} \longleftrightarrow \frac{rect(\frac{\omega}{8\pi})}{4} \\
    \textrm{By linearity: } \frac{4sin(4\pi t)}{4\pi t} \longleftrightarrow rect(\frac{\omega}{8\pi}) = X_1(j\omega) \\ \\
    x_2(t) = cos(2\pi t) = \frac{1}{2} e^{j2\pi t} + \frac{1}{2} e^{-j2\pi t} \\
    \mathscr{F}\{\frac{1}{2} e^{j2\pi t} + \frac{1}{2} e^{-j2\pi t}\} \longleftrightarrow \pi [\delta(\omega+2\pi) + \delta(\omega-2\pi)] = X_2(j\omega) \\ \\
    \textrm{By modulation property of CTFT: } \\
    x(t) = x_1(t) x_2(t) \longleftrightarrow X(j\omega) = \frac{1}{2\pi} X_1(j\omega)*X_2(j\omega) \\
    X(j\omega) = \frac{1}{2\pi} rect(\frac{\omega}{8\pi})*\pi [\delta(\omega+2\pi) + \delta(\omega-2\pi)] = \frac{1}{2} [rect(\frac{\omega}{8\pi})*\delta(\omega+2\pi) + rect(\frac{\omega}{8\pi})*\delta(\omega-2\pi)] \\
    X(j\omega) = \frac{1}{2} rect(\frac{\omega+2\pi}{8\pi}) + \frac{1}{2} rect(\frac{\omega-2\pi}{8\pi}) 
    \end{math}
    \subsubsection*{ii)}
    \begin{math}
    y(t) = h(t)*x(t) \\
    \textrm{By convolution property of CTFT: } y(t) = h(t) * x(t) \longleftrightarrow Y(j\omega) = H(j\omega)X(j\omega)\\
    Y(j\omega) = \left(1-rect(\frac{\omega}{4\pi})\right)  \left(\frac{1}{2} rect(\frac{\omega-2\pi}{8\pi}) + \frac{1}{2} rect(\frac{\omega+2\pi}{8\pi})\right) \\
    Y(j\omega) = \frac{1}{2} rect(\frac{\omega-2\pi}{8\pi}) + \frac{1}{2} rect(\frac{\omega+2\pi}{8\pi}) - \frac{1}{2} \left(rect(\frac{\omega}{4\pi}) rect(\frac{\omega-2\pi}{8\pi})+ rect(\frac{\omega}{4\pi}) rect(\frac{\omega+2\pi}{8\pi})\right) \\
    Y(j\omega) = rect(\frac{\omega+4\pi}{4\pi}) + rect(\frac{\omega-4\pi}{4\pi})
    \end{math}
    \subsection*{b)}
    \subsubsection*{i)}
    By modulation property of CTFT: \\
    \begin{math}
    z(t) \longleftrightarrow Z(j\omega) = \frac{1}{2\pi} Y(j\omega) * \mathscr{F}\{\frac{sin(2\pi t)}{\pi t}\} =  \frac{1}{2\pi} Y(j\omega) * rect(\frac{\omega}{4\pi}) \\ \\ 
    Y(j\omega) = \frac{1}{\pi} rect(\frac{\omega}{4\pi}) * \pi [\delta(\omega-4\pi) + \delta(\omega+4\pi)] \\
    \Rightarrow Z(j\omega) = \frac{1}{2\pi} \left(rect(\frac{\omega}{4\pi})* rect(\frac{\omega}{4\pi})\right) * [\delta(\omega-4\pi) + \delta(\omega+4\pi)]
    \end{math} 
    \subsubsection*{ii)}
    \begin{math}
    y(t) \longleftrightarrow Y(j\omega) = rect(\frac{\omega+4\pi}{4\pi}) + rect(\frac{\omega-4\pi}{4\pi}) \\
    Y(j\omega) = \frac{1}{\pi} rect(\frac{\omega}{4\pi}) * \pi [\delta(\omega-4\pi) + \delta(\omega+4\pi)] \\
    \textrm{By modulation property of CTFT: } \\
    \mathscr{F}^{-1}\{Y(j\omega)\} = y(t) =  \frac{2sin(2\pi t)}{\pi t} cos(4\pi t) 
    \end{math} 
    \section*{Question 4}
    \subsection*{a)}
    \subsubsection*{i)}
    \begin{math} 
    X(e^{j\Omega}) = \displaystyle\sum_{n=-\infty}^{\infty} \underbrace{\delta[n] e^{-j\Omega n}}_{\delta[n] e^{-j\Omega 0}} = \displaystyle\sum_{n=-\infty}^{\infty} \delta[n] = 1 
    \end{math} 
     \subsubsection*{ii)}
    \begin{math} 
    X(e^{j\Omega}) = \displaystyle\sum_{n=-\infty}^{\infty} (2\delta[n-3] - \delta[n-10] ) e^{-j\Omega n} \stackrel{\text{(by linearity)}}{=} 2\displaystyle\sum_{n=-\infty}^{\infty} \delta[n-3] e^{-j\Omega n} - \displaystyle\sum_{n=-\infty}^{\infty} \delta[n-10] e^{-j\Omega n} \\ \\
    \textrm{By time-shift property of DTFT: } \\ 
    X(e^{j\Omega}) = 2e^{-j3\Omega} - e^{-j10\Omega} 
    \end{math} 
      \subsubsection*{iii)}
    \begin{math} 
    X(e^{j\Omega}) = \displaystyle\sum_{n=-\infty}^{\infty} x[n] e^{-j\Omega n} = \displaystyle\sum_{n=1}^{4} \frac{1}{n^{2}} e^{-j\Omega n} = e^{-j\Omega} + \frac{1}{4} e^{-j2\Omega} +\frac{1}{9} e^{-j3\Omega} + \frac{1}{16} e^{-j4\Omega}
    \end{math} 
     \subsubsection*{iv)}
    \begin{math}
    X(e^{j\Omega}) = \displaystyle\sum_{n=-\infty}^{\infty} \left(\left(\frac{1}{2}\right)^{n} u[n] - 3^{n} u[-n-1]\right)  e^{-j\Omega n} \\ 
    \stackrel{\text{(by linearity)}}{=}  \displaystyle\sum_{n=-\infty}^{\infty} \left(\frac{1}{2}\right)^{n} u[n] e^{-j\Omega n} - \displaystyle\sum_{n=-\infty}^{\infty} 3^{n} u[-n-1] e^{-j\Omega n} = \displaystyle\sum_{n=0}^{\infty} \left(\frac{1}{2} e^{-j\Omega}\right)^{n} - \displaystyle\sum_{n=-\infty}^{-1} 3^{n} e^{-j\Omega n} \\
    \displaystyle\sum_{n=0}^{\infty} \left(\frac{1}{2} e^{-j\Omega}\right)^{n} = \frac{1}{1- \frac{1}{2} e^{-j\Omega}} \quad \left(\textrm{since } \left\vert\frac{1}{2} e^{-j\Omega}\right\vert = \frac{1}{2} < 1, \textrm{so the expression is convergent}\right) \\ 
    \textrm{Let } m = -n: \displaystyle\sum_{n=-\infty}^{-1} 3^{n} e^{-j\Omega n} = \displaystyle\sum_{m=1}^{\infty} 3^{-m} e^{j\Omega m} = \displaystyle\sum_{m=1}^{\infty} \left(\frac{1}{3} e^{j\Omega}\right)^{m} = \underbrace{\left[\displaystyle\sum_{m=0}^{\infty} \left(\frac{1}{3} e^{j\Omega}\right)^{m}\right]}_{\frac{1}{1- \frac{1}{3} e^{j\Omega}}} -1 \\
    \Rightarrow X(e^{j\Omega}) = \frac{1}{1- \frac{1}{2} e^{-j\Omega}} - \left(\frac{1}{1- \frac{1}{3} e^{j\Omega}} -1\right)
    \end{math} 
    \subsubsection*{v)}
    Say that,  \begin{math} 
    \hat x[n] = \left(\frac{1}{2}\right)^{n} u[n] - 3^{n} u[-n-1]  \textrm{ and } \mathscr{F}\{\hat x[n]\}= \hat X(e^{j\Omega}) =  \frac{1}{1- \frac{1}{2} e^{-j\Omega}} - \left(\frac{1}{1- \frac{1}{3} e^{j\Omega}} -1\right)
    \end{math} \\ 
    By time-shift property of DTFT: \\ 
    \begin{math} 
    x[n] = \hat x[n-7] \longleftrightarrow X(e^{j\Omega}) = \hat X(e^{j\Omega}) e^{-j7\Omega}  \\ \\
    \Rightarrow X(e^{j\Omega}) = \frac{e^{-j7\Omega}}{1- \frac{1}{2} e^{-j\Omega}} - \left(\frac{e^{-j7\Omega}}{1- \frac{1}{3} e^{j\Omega}} -e^{-j7\Omega}\right)
    \end{math} 
    \subsubsection*{vi)}
    Let x[n] be a periodic signal with fundamental period N. Then, \\
    \begin{math}
    \mathscr{F}\{x[n]\} = X(e^{j\Omega}) = \displaystyle\sum_{m=-\infty}^{\infty} \displaystyle\sum_{k=k_0}^{k_0 + N-1} a_k 2 \pi \delta(\Omega -k \frac{2\pi}{N} - 2\pi m) \end{math} \\
    Note that for this signal if we consider the interval \(0 \leq \Omega < 2\pi \), DTFT of the signal will be written as: \\
    \begin{math}
    \mathscr{F}\{x[n]\} = X(e^{j\Omega}) = \displaystyle\sum_{k=0}^{2} a_k 2 \pi \delta(\Omega -k \frac{2\pi}{3}) \\
    \textrm{First, find the DTFS coefficients of x[n]: }  a_k = \frac{1}{3} \displaystyle\sum_{n=0}^{2} x[n] e^{-jk\frac{2\pi}{3}n}\\ 
    \Rightarrow a_0 = \frac{1}{3} \displaystyle\sum_{n=0}^{2} x[n] = \frac{1}{3} + \frac{1}{3} = \frac{2}{3} \\
    \Rightarrow a_1 = \frac{1}{3} \displaystyle\sum_{n=0}^{2} x[n] e^{-j\frac{2\pi}{3}n} = \frac{1}{3} e^{-j\frac{2\pi}{3}} + \frac{1}{3} e^{-j\frac{4\pi}{3}} =  \frac{-1}{3} \\ 
    \Rightarrow a_2 = \frac{1}{3} \displaystyle\sum_{n=0}^{2} x[n] e^{-j\frac{4\pi}{3}n} = \frac{1}{3} e^{-j\frac{4\pi}{3}} + \frac{1}{3} e^{-j\frac{8\pi}{3}} =  \frac{-1}{3} \\ \\   
    X(e^{j\Omega}) =  \displaystyle\sum_{k=0}^{2} a_k 2\pi \delta(\Omega -k \frac{2\pi}{3}) = a_0 2\pi \delta(\Omega) + a_1 2\pi \delta(\Omega - \frac{2\pi}{3}) + a_2 2\pi \delta(\Omega - \frac{4\pi}{3}) \\ 
     X(e^{j\Omega}) = \frac{2\pi}{3} \left(2 \delta(\Omega) - \delta(\Omega - \frac{2\pi}{3}) - \delta(\Omega - \frac{4\pi}{3})\right)
    \end{math} 
    
    \subsection*{b)}
    \subsubsection*{i)}
    \begin{math}
    x[n] = \frac{1}{2\pi} \displaystyle\int_{\Omega_0}^{\Omega_0 + 2\pi} \underbrace{X(e^{j\Omega})}_{=1} e^{j\Omega n} d\Omega = \frac{1}{2\pi} \displaystyle\int_{-\pi}^{\pi} e^{j\Omega n} d\Omega = \frac{1}{2\pi j n} \left(e^{j\pi n} - e^{-j\pi n}\right) \\
    x[n] = \frac{1}{\pi n} \frac{1}{2j} \left(e^{j\pi n} - e^{-j\pi n}\right) = \frac{sin(\pi n)}{\pi n} \qquad \boxed{\textrm{Recall that, } sinc(t) = \begin{cases}
        \frac{sin(\pi t)}{\pi t},& t \neq 0 \\
        1,& t=0\\
      \end{cases}}\\
      \textrm{Here, n is an integer. So, } \frac{sin(\pi n)}{\pi n} = \begin{cases}
        1,& n=0 \\
        0,& otherwise\\
      \end{cases} \Rightarrow x[n] = \delta[n] 
    \end{math}
    \subsubsection*{ii)}
    \begin{math}
    \textrm{We know that, } \mathscr{F}\{1\} =  2\pi \displaystyle\sum_{m=-\infty}^{\infty} \delta(\Omega - 2\pi m) \\
    \textrm{By frequency-shift property of DTFT: } \\
     \mathscr{F}\{e^{j\Omega_0 n}\} = 2\pi \displaystyle\sum_{m=-\infty}^{\infty} \delta(\Omega - \Omega_0 - 2\pi m) \\ 
    \Rightarrow \mathscr{F}\{e^{j\frac{\pi}{3}n}\} = 2\pi \displaystyle\sum_{m=-\infty}^{\infty} \delta(\Omega - \frac{\pi}{3} - 2\pi m) = 2\pi X(e^{j\Omega}) \\
    \Rightarrow \mathscr{F}^{-1}\{\displaystyle\sum_{m=-\infty}^{\infty} \delta(\Omega - \frac{\pi}{3} - 2\pi m)\} = \frac{e^{j\frac{\pi}{3}n}}{2\pi} = x[n]  
    \end{math}
    \subsubsection*{iii)}
    \begin{math}
    x[n] = \frac{1}{2\pi} \displaystyle\int_{-\pi}^{\pi} X(e^{j\Omega}) e^{j\Omega n} d\Omega = \frac{1}{2\pi} \underbrace{\displaystyle\int_{-\pi}^{\frac{-\pi}{2}} e^{j\Omega n} d\Omega}_{= \displaystyle\int_{\frac{\pi}{2}}^{\pi} e^{-j\Omega n} d\Omega} + \frac{1}{2\pi} \displaystyle\int_{\frac{\pi}{2}}^{\pi} e^{j\Omega n} d\Omega = \frac{1}{2\pi} \displaystyle\int_{\frac{\pi}{2}}^{\pi} e^{j\Omega n} + e^{-j\Omega n} d\Omega \\
    x[n] = \frac{1}{\pi} \displaystyle\int_{\frac{\pi}{2}}^{\pi} cos(\Omega n) = \frac{1}{\pi n} sin(\Omega n)\Big|_\frac{\pi}{2}^{\pi} = \frac{sin(\pi n)}{\pi n} - \frac{sin(\frac{\pi}{2} n)}{\pi n}
    \end{math}
    \subsubsection*{iv)}
    By frequency-shift property of DTFT: \\ 
    \begin{math}
    x[n] \longleftrightarrow X(e^{j\Omega}) \\
    x[n] e^{j\frac{\pi}{4}n} \longleftrightarrow X(e^{j(\Omega-\frac{\pi}{4})}) = Y(e^{j\Omega})\\ \\
    \textrm{Therefore, } y[n] = x[n] e^{j\frac{\pi}{4}n} = \frac{sin(\pi n)e^{j\frac{\pi}{4}n}}{\pi n} - \frac{sin(\frac{\pi}{2} n)e^{j\frac{\pi}{4}n}}{\pi n}
    \end{math}
    \section*{Question 5}
    \subsection*{a)}
    \subsection*{b)}
    \subsection*{c)}
    \subsection*{d)}
    \section*{Question 6}

\end{document}