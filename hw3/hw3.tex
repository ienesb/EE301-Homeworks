\documentclass[12pt]{article}
\usepackage[utf8]{inputenc, }
\usepackage{graphicx}
\usepackage{hyperref}
\usepackage[margin=1in]{geometry}
\usepackage{setspace}
\usepackage{color}
\usepackage{pdfpages}
\usepackage{amsmath}
\usepackage{amsfonts}
\usepackage{float}
\usepackage{tikz}
\usepackage{pgfplots}
\usepackage{enumitem}
\usepackage{xpatch}
\usepackage{svg}
\usepackage{mathrsfs}
% \usepackage[nocheck]{fancyhdr}

\sloppy
\definecolor{lightgray}{gray}{0.5}
\setlength{\parindent}{0pt}

\hypersetup{
    colorlinks,
    citecolor=black,
    filecolor=black,
    linkcolor=black,
    urlcolor=black
    pdftitle={EE300 İsmail Enes Bülbül}
}
\onehalfspacing

% \raggedright

\title{EE301 Homework-3}
\author{İsmail Enes Bülbül, Eren Meydanlı, Ahmet Caner Akar}
%\date{October 2022}
\renewcommand*\contentsname{Table of Contents}
\renewcommand*{\refname}{}
% \fancyhf{} % sets both header and footer to nothing
% \renewcommand{\headrulewidth}{0pt}
\begin{document}

\maketitle
% \tableofcontents
% \newpage

    \section*{Question 1}
    \subsection*{a)}
    \subsection*{b)}
    \section*{Question 2}
    \subsection*{a)}
    \subsection*{b)}
    \subsection*{c)}
    \section*{Question 3}
    \subsection*{a)}
    \subsection*{b)}
    \section*{Question 4}
    \subsection*{a)}
    \subsubsection*{i)}
    \begin{math} 
    X(e^{j\Omega}) = \displaystyle\sum_{n=-\infty}^{\infty} \underbrace{\delta[n] e^{-j\Omega n}}_{\delta[n] e^{-j\Omega 0}} = \displaystyle\sum_{n=-\infty}^{\infty} \delta[n] = 1 
    \end{math} 
     \subsubsection*{ii)}
    \begin{math} 
    X(e^{j\Omega}) = \displaystyle\sum_{n=-\infty}^{\infty} (2\delta[n-3] - \delta[n-10] ) e^{-j\Omega n} \stackrel{\text{(by linearity)}}{=} 2\displaystyle\sum_{n=-\infty}^{\infty} \delta[n-3] e^{-j\Omega n} - \displaystyle\sum_{n=-\infty}^{\infty} \delta[n-10] e^{-j\Omega n} \\ \\
    \textrm{By time-shifting property of DTFT: } \\ 
    X(e^{j\Omega}) = 2e^{-j3\Omega} - e^{-j10\Omega} 
    \end{math} 
      \subsubsection*{iii)}
    \begin{math} 
    X(e^{j\Omega}) = \displaystyle\sum_{n=-\infty}^{\infty} x[n] e^{-j\Omega n} = \displaystyle\sum_{n=1}^{4} \frac{1}{n^{2}} e^{-j\Omega n} = e^{-j\Omega} + \frac{1}{4} e^{-j2\Omega} +\frac{1}{9} e^{-j3\Omega} + \frac{1}{16} e^{-j4\Omega}
    \end{math} 
     \subsubsection*{iv)}
    \begin{math}
    X(e^{j\Omega}) = \displaystyle\sum_{n=-\infty}^{\infty} \left(\left(\frac{1}{2}\right)^{n} u[n] - 3^{n} u[-n-1]\right)  e^{-j\Omega n} \\ 
    \stackrel{\text{(by linearity)}}{=}  \displaystyle\sum_{n=-\infty}^{\infty} \left(\frac{1}{2}\right)^{n} u[n] e^{-j\Omega n} - \displaystyle\sum_{n=-\infty}^{\infty} 3^{n} u[-n-1] e^{-j\Omega n} = \displaystyle\sum_{n=0}^{\infty} \left(\frac{1}{2} e^{-j\Omega}\right)^{n} - \displaystyle\sum_{n=-\infty}^{-1} 3^{n} e^{-j\Omega n} \\
    \displaystyle\sum_{n=0}^{\infty} \left(\frac{1}{2} e^{-j\Omega}\right)^{n} = \frac{1}{1- \frac{1}{2} e^{-j\Omega}} \quad \left(\textrm{since } \left\vert\frac{1}{2} e^{-j\Omega}\right\vert = \frac{1}{2} < 1, \textrm{so the expression is convergent}\right) \\ 
    \textrm{Let } m = -n: \displaystyle\sum_{n=-\infty}^{-1} 3^{n} e^{-j\Omega n} = \displaystyle\sum_{m=1}^{\infty} 3^{-m} e^{j\Omega m} = \displaystyle\sum_{m=1}^{\infty} \left(\frac{1}{3} e^{j\Omega}\right)^{m} = \underbrace{\left[\displaystyle\sum_{m=0}^{\infty} \left(\frac{1}{3} e^{j\Omega}\right)^{m}\right]}_{\frac{1}{1- \frac{1}{3} e^{j\Omega}}} -1 \\
    \Rightarrow X(e^{j\Omega}) = \frac{1}{1- \frac{1}{2} e^{-j\Omega}} - \left(\frac{1}{1- \frac{1}{3} e^{j\Omega}} -1\right)
    \end{math} 
    \subsubsection*{v)}
    Say that,  \begin{math} 
    \hat x[n] = \left(\frac{1}{2}\right)^{n} u[n] - 3^{n} u[-n-1]  \textrm{ and } \mathscr{F}\{\hat x[n]\}= \hat X(e^{j\Omega}) =  \frac{1}{1- \frac{1}{2} e^{-j\Omega}} - \left(\frac{1}{1- \frac{1}{3} e^{j\Omega}} -1\right)
    \end{math} \\ 
    By time-shifting property of DTFT: \\ 
    \begin{math} 
    x[n] = \hat x[n-7] \longleftrightarrow X(e^{j\Omega}) = \hat X(e^{j\Omega}) e^{-j7\Omega}  \\ \\
    \Rightarrow X(e^{j\Omega}) = \frac{e^{-j7\Omega}}{1- \frac{1}{2} e^{-j\Omega}} - \left(\frac{e^{-j7\Omega}}{1- \frac{1}{3} e^{j\Omega}} -e^{-j7\Omega}\right)
    \end{math} 
    \subsubsection*{vi)}
    \subsection*{b)}
    \section*{Question 5}
    \subsection*{a)}
    \subsection*{b)}
    \subsection*{c)}
    \subsection*{d)}
    \section*{Question 6}

\end{document}