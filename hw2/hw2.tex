\documentclass[12pt]{article}
\usepackage[utf8]{inputenc, }
\usepackage{graphicx}
\usepackage{hyperref}
\usepackage[margin=1in]{geometry}
\usepackage{setspace}
\usepackage{color}
\usepackage{pdfpages}
\usepackage{amsmath}
\usepackage{amsfonts}
\usepackage{float}
\usepackage{tikz}
\usepackage{pgfplots}
\usepackage{enumitem}
\usepackage{xpatch}
\usepackage{svg}
% \usepackage[nocheck]{fancyhdr}

\sloppy
\definecolor{lightgray}{gray}{0.5}
\setlength{\parindent}{0pt}

\hypersetup{
    colorlinks,
    citecolor=black,
    filecolor=black,
    linkcolor=black,
    urlcolor=black
    pdftitle={EE300 İsmail Enes Bülbül}
}
\onehalfspacing

% \raggedright

\title{EE301 Homework-2}
\author{İsmail Enes Bülbül, Eren Meydanlı, Ahmet Caner Akar}
%\date{October 2022}
\renewcommand*\contentsname{Table of Contents}
\renewcommand*{\refname}{}
% \fancyhf{} % sets both header and footer to nothing
% \renewcommand{\headrulewidth}{0pt}
\begin{document}

\maketitle
% \tableofcontents
% \newpage


    \section*{Question 1}
    \subsection*{a)}
    Let \(x[n] = \delta[n]\). Then \(y[n] = h[n]\) and equation becomes:\\
    \begin{math}
      h[n] - ah[n-1] = \delta[n] - b\delta[n-1]
    \end{math}\\
    Also, we know that \(h[n] = 0\) for \(n < 0\) since the system is said to be causal.\\ \\
    For \(n = 0\), we have:\\
    \begin{math}
      h[0] - ah[-1] = \delta[0] - b\delta[-1]\\
      h[0] = 1
    \end{math}\\ \\
    For \(n = 1\), we have:\\
    \begin{math}
      h[1] - ah[0] = \delta[1] - b\delta[0]\\
      h[1] - a = -b\\
      h[1] = a - b
    \end{math}\\ \\
    For \(n = 2\), we have:\\
    \begin{math}
      h[2] - ah[1] = \delta[2] - b\delta[1]\\
      h[2] - a(a-b) = 0\\
      h[2] = a(a-b)
    \end{math}\\ \\
    For \(n = 3\), we have:\\
    \begin{math}
      h[3] - ah[2] = \delta[3] - b\delta[2]\\
      h[3] - a^2(a-b) = 0\\
      h[3] = a^2(a-b)
    \end{math}\\ \\
    In general, for \(n > 0\), we have \(h[n] = a^{n-1}(a-b)\).\\
    \begin{math}
      h[n] = \begin{cases}
        a^{n-1}(a-b),& n>0 \\
        1,& n=0\\
        0,& n<0\\
      \end{cases}
    \end{math}\\ \\
    If the system is stable, then the following expressions must be satisfied: \\
    \begin{math}
      \displaystyle\sum_{k = -\infty}^{\infty}|h[k]| < \infty \Longleftrightarrow 1 + \displaystyle\sum_{k = 1}^{\infty}|a|^{k-1}|a-b| < \infty \Longleftrightarrow 1 + |a-b|\displaystyle\sum_{k = 0}^{\infty}|a|^k < \infty\\ \\ 
    \end{math}
    Therefore, the system is said to be stable \begin{math} \Longleftrightarrow \begin{cases}
        a \in \mathbb{R},& \textrm{if }a=b \\
        -1<a<1,& \textrm{otherwise}\\
      \end{cases} \end{math}
    \subsection*{b)}
    \begin{math} y[n] = h[n]*x[n]  \quad \textrm{and} \quad x[n] = e^{j\Omega_1 n} + e^{j\Omega_2 n} \\
    \textrm{By distribution property of the convolution over addition:}\\
    y[n] = h[n] * e^{j\Omega_1 n} + h[n] * e^{j\Omega_2 n} = \displaystyle\sum_{k=-\infty}^{\infty} h[k] e^{j\Omega_1 (n-k)} + \displaystyle\sum_{k=-\infty}^{\infty} h[k] e^{j\Omega_2 (n-k)}  \\ 
    y[n] = e^{j\Omega_1 n} + \displaystyle\sum_{k=1}^{\infty} a^{k-1} (a-\frac{1}{a}) e^{j\Omega_1 (n-k)} + e^{j\Omega_2 n} + \displaystyle\sum_{k=1}^{\infty} a^{k-1} (a-\frac{1}{a}) e^{j\Omega_2 (n-k)} \\ 
    y[n] = e^{j\Omega_1 n} + \displaystyle\sum_{k=1}^{\infty} a^{k} \frac{(a^2-1)}{a^2} e^{j\Omega_1 (n-k)} + e^{j\Omega_2 n} + \displaystyle\sum_{k=1}^{\infty} a^{k} \frac{(a^2-1)}{a^2} e^{j\Omega_2 (n-k)} \\ 
    y[n] = e^{j\Omega_1 n} + \frac{(a^2 -1)}{a^2} e^{j\Omega_1 n} \displaystyle\sum_{k=1}^{\infty} a^{k} e^{-j\Omega_1 k}+ e^{j\Omega_2 n} + \frac{(a^2 -1)}{a^2} e^{j\Omega_2 n} \displaystyle\sum_{k=1}^{\infty} a^{k} e^{-j\Omega_2 k} \\
    y[n] = e^{j\Omega_1 n} + \frac{(a^2 -1)}{a^2} e^{j\Omega_1 n} \displaystyle\sum_{k=1}^{\infty} (ae^{-j\Omega_1 })^k + e^{j\Omega_2 n} + \frac{(a^2 -1)}{a^2} e^{j\Omega_2 n} \displaystyle\sum_{k=1}^{\infty} (ae^{-j\Omega_2 })^k \\
    \end{math} 
    We know that \(-1\leq a \leq 1\) if \(a=b\) for a stable system. Then we have \(|ae^{-j\Omega}| \leq 1\) for any \(\Omega\).\\ \\
    \begin{math}
      y[n] = e^{j\Omega_1 n} + \frac{(a^2 -1)}{a^2} e^{j\Omega_1 n} \frac{ae^{-j\Omega_1}}{1-ae^{-j\Omega_1}} + e^{j\Omega_2 n} + \frac{(a^2 -1)}{a^2} e^{j\Omega_2 n} \frac{ae^{-j\Omega_2}}{1-ae^{-j\Omega_2}} \\
    \end{math} 
    \section*{Question 2}
    \subsection*{a)}
	\(x(t)\) and \(h(t)\) can be written as: \\
    \begin{math}
      x(t) = u(t+1)-u(t-1) \\
      h(t) =  (1-t)[u(t)-u(t-1)] = \begin{cases}
      1-t,& 0<t<1 \\
      0,& otherwise
    \end{cases} \end{math}\\
      Then, \begin{math}y(t) = x(t)*h(t) \end{math}\\ \\
      Before evaluating the convolution, consider the following: \\
     \begin{math} \hat{y}(t) = u(t)*h(t) = \int_{-\infty}^{\infty} u(t-\tau)h(\tau) \,d\tau \end{math} \\
     \begin{math}  = \int_{-\infty}^{t} h(\tau) \,d\tau = \begin{cases}
      0,& t < 0\\
      \int_{0}^{t} (1-\tau) \,d\tau,& 0 < t < 1\\
     \int_{0}^{1} (1-\tau) \,d\tau,& t > 1
    \end{cases} \ =   \begin{cases}
      0,& t < 0\\
      t - \frac{t^{2}}{2} ,& 0 < t < 1\\
     \frac{1}{2},& t > 1
    \end{cases}  \end{math} \\ \\
    By the properties of the LTI system: \\ 
    \begin{math}
      \resizebox{.9\hsize}{!}{$
      \hat{y}(t+1) = u(t+1)*h(t) = \begin{cases}
      0,& t+1 < 0\\
      (t+1) - \frac{(t+1)^{2}}{2},& 0 < t+1 <1\\
     \frac{1}{2},& t+1 > 1
    \end{cases} \ =  \begin{cases}
      0,& t < -1\\
      (t+1) - \frac{(t+1)^{2}}{2},& -1 < t < 0\\
     \frac{1}{2},& t > 0
    \end{cases}$}  \\ \\ \\
    \resizebox{.9\hsize}{!}{$
    \hat{y}(t-1) = u(t-1)*h(t) = \begin{cases}
      0,& t-1 < 0\\
      (t-1) - \frac{(t-1)^{2}}{2},& 0 < t-1 <1\\
     \frac{1}{2},& t-1 > 1
    \end{cases} \ =  \begin{cases}
      0,& t < 1\\
      (t-1) - \frac{(t-1)^{2}}{2},& 1 < t < 2\\
     \frac{1}{2},& t > 2
    \end{cases}$} \end{math}\\ \\ \\
    Therefore, \begin{math} y(t) = \hat{y}(t+1) - \hat{y}(t-1) =  \begin{cases}
      0,& t < -1\\
      \frac{1-t^{2}}{2},& -1 < t < 0\\
     \frac{1}{2},& 0 < t < 1\\
     \frac{1}{2} - (\frac{t^{2}-4t+3}{2}),& 1 < t < 2\\
     0,& t > 2
    \end{cases}  \end{math} \\ 
    Also, \(y(t)\) can be written as: \\ \\
    \begin{math} y(t) = (\frac{1-t^2}{2})\cdot u(t+1) + (\frac{t^{2}}{2})\cdot u(t) - (\frac{t^2-4t+3}{2})\cdot u(t-1) + (\frac{t^2-4t+2}{2})\cdot u(t-2)\end{math} 

    \subsection*{b)} \begin{math}
    w(t) = h(t)*g(t) \\ 
    g(t)\end{math} can be written as: \begin{math} g(t) = x(t) + x(t-1) - x(t+1) \end{math} \\
    Then, \begin{math} w(t) = h(t)*[x(t) + x(t-1) - x(t+1)] \end{math} \\ \\
    By the distributive property of the convolution over addition: \\
    \begin{math} w(t) = h(t)*x(t) + h(t)*x(t-1) - h(t)*x(t+1)\\
    w(t) = y(t) + y(t-1) - y(t+1) \end{math} [by considering part a] \\ \\ 
    By the time-invarince property of the LTI system: \\
   \begin{math} 
    y(t-1) = (\frac{2t-t^2}{2})\cdot u(t) + (\frac{t^{2}-2t+1}{2})\cdot u(t-1) - (\frac{t^2-6t+8}{2})\cdot u(t-2) + (\frac{t^2-6t+7}{2})\cdot u(t-3) \\
    y(t+1) = (\frac{-t^2-2t}{2})\cdot u(t+2) + (\frac{t^{2}+2t+1}{2})\cdot u(t+1) - (\frac{t^2-2t}{2})\cdot u(t) + (\frac{t^2-2t-1}{2})\cdot u(t-1) \\ \\ 
    \textrm{As a result:} \\ 
    w(t) = (\frac{-t^2-2t}{2})\cdot u(t+2) + (-t^2-t)\cdot u(t+1) + (\frac{t^2}{2})\cdot u(t) + (\frac{-t^2+4t-1}{2})\cdot u(t-1) + (t-3)\cdot u(t-2) + (\frac{t^2-6t+7}{2})\cdot u(t-3)
    \end{math}
				   	 
    \section*{Question 3}
    \subsection*{a)}
    \begin{math} x(t) = \ \cdots + \delta(t+2)-2\delta(t+1) +\delta(t) - 2\delta(t-1)+\delta(t-2)-2\delta(t-3)+ \ \cdots  \\ 
    \textrm{It is a perodic signal with fundamental period \(T_0 = 2\).} \\
    a_k = \frac{1}{T_0} \displaystyle\int_{t_o}^{t_0+T_0}x(t) e^{-jk\omega_0t}dt = \frac{1}{2} \displaystyle\int_{\frac{-1}{2}}^{\frac{3}{2}}x(t) e^{-jk\pi t}dt\\ a_k =  \frac{1}{2} \underbrace{\displaystyle\int_{0^{-}}^{0^{+}}\delta(t) e^{-jk\pi t}dt}_{e^{-jk\pi(0)}} + \frac{1}{2} \underbrace{\displaystyle\int_{1^{-}}^{1^{+}}-2\delta(t-1) e^{-jk\pi t}dt}_{e^{-jk\pi(1)}} \quad \
    \boxed{\textrm{Also, recall that } \displaystyle\int_{-\infty}^{\infty}\delta(t-t_0) x(t) = x(t_0) } \\ \\
   \Rightarrow a_k = \frac{1}{2} - e^{-jk\pi} \\ 
   \Rightarrow x(t) = \displaystyle\sum_{k=-\infty}^{\infty} a_k  e^{jk\omega_0 t} = \displaystyle\sum_{k=-\infty}^{\infty} \left(\frac{1}{2} - e^{-jk\pi}\right)  e^{jk\pi t}\\
    \end{math}
    \subsection*{b)}
    \(x(t)\) is a perodic signal with fundamental period, \(T_0 = 4\). \\
    \begin{math}  
    a_k = \frac{1}{4} \displaystyle\int_{-2}^{2} x(t) e^{-jk\omega_0 t} dt = \frac{1}{4} \displaystyle\int_{0}^{1} cos(\frac{\pi t}{2}) e^{-jk\frac{\pi}{2} t} dt \\ \\
         a_k =  \frac{1}{4} \displaystyle\int_{0}^{1} \frac{1}{2} \left(e^{j\frac{\pi}{2}t} + e^{-j\frac{\pi}{2}t}\right) e^{-jk\frac{\pi}{2} t} dt = \frac{1}{8} \displaystyle\int_{0}^{1} \left(e^{-j\frac{\pi}{2}(k-1)t} + e^{-j\frac{\pi}{2}(k+1)t}\right) dt \\ \\
     a_k = - \frac{1}{8}  \left(\frac{e^{-j\frac{\pi}{2}(k-1)t}}{j(k-1)\frac{\pi}{2}}  \displaystyle\Big|_{0}^{1} + \frac{e^{-j\frac{\pi}{2}(k+1)t}}{j(k+1)\frac{\pi}{2}}  \displaystyle\Big|_{0}^{1}\right) = \frac{1}{8} \left[\frac{1-e^{-j\frac{\pi}{2}(k-1)}}{j(k-1)\frac{\pi}{2}}\right] + \frac{1}{8} \left[\frac{1-e^{-j\frac{\pi}{2}(k+1)}}{j(k+1)\frac{\pi}{2}}\right] \\ \\
    a_k = \frac{1}{8} \left[\frac{1-e^{-j\frac{\pi}{2}k}e^{j\frac{\pi}{2}}}{j(k-1)\frac{\pi}{2}}\right] + \frac{1}{8} \left[\frac{1-e^{-j\frac{\pi}{2}k}e^{-j\frac{\pi}{2}}}{j(k+1)\frac{\pi}{2}}\right] \\ \\ 
 a_k = \frac{1}{8} \left[\frac{1-je^{-j\frac{\pi}{2}k}}{j(k-1)\frac{\pi}{2}}\right] + \frac{1}{8} \left[\frac{1+je^{-j\frac{\pi}{2}k}}{j(k+1)\frac{\pi}{2}}\right] = \frac{1}{4} \left[\frac{k-je^{-j\frac{\pi}{2}k}}{j(k^{2} -1)\frac{\pi}{2}}\right] = -\frac{1}{4} \left[\frac{e^{-j\frac{\pi}{2}k}+jk}{(k^{2} -1)\frac{\pi}{2}}\right]   \\ \\ 
   \Rightarrow x(t) = \displaystyle\sum_{k=-\infty}^{\infty} a_k  e^{jk\omega_0 t} = \displaystyle\sum_{k=-\infty}^{\infty} -\frac{1}{4}\left(\frac{e^{-j\frac{\pi}{2}k}+jk}{(k^{2} -1)\frac{\pi}{2}}\right) e^{jk\frac{\pi}{2} t}
    \end{math}
    
    \subsection*{c)}
    \begin{math} x[n] = (-1)^{n} + j^{n} + cos(\frac{2\pi n}{3}) = e^{j\pi n }+ e^{j\frac{\pi}{2}n} + cos(\frac{2\pi n}{3}) \\ 
    \textrm{Therefore,} \ x[n] \ \textrm{can be written as a summation of 3 distinct perodic signals: } \\
    x[n] = \underbrace{e^{j\pi n }}_{x_1[n]}+ \underbrace{e^{j\frac{\pi}{2}n}}_{x_2[n]} + \underbrace{cos\left(\frac{2\pi n}{3}\right)}_{x_3[n]} = x_1[n] +x_2[n] +x_3[n] \end{math}\\ \\
    \(x_1[n]\) \ is a periodic signal with fundamental period \(N_0 = 2\). \\   
    \(x_2[n]\) \ is a periodic signal with fundamental period \(N_0 = 4\). \\
    \(x_3[n]\) \ is a periodic signal with fundamental period \(N_0 = 3\). \\ \\
    Then, the fundamental period of \(x[n]\) is the least common multiple of the fundamental periods of these three signals, so \(N_0 = 12\). \\
    \begin{math} a_k = \frac{1}{T_0} \displaystyle\sum_{n=n_0}^{n_0+N_0-1}x[n] e^{-jk\frac{2\pi}{N_0}n} = \frac{1}{12} \displaystyle\sum_{n=-6}^{5}x[n] e^{-jk\frac{\pi}{6}n} \end{math}
    \subsection*{d)}
     \(x(t)\) is periodic signal with fundamental period \(T_0 = 1\). \\
     \begin{math} a_k = \frac{1}{T_0} \displaystyle\int_{t_o}^{t_0+T_0}x(t) e^{-jk\omega_0t}dt = \displaystyle\int_{0}^{1}e^{-t} e^{-jk2\pi t}dt = \displaystyle\int_{0}^{1}e^{-t(1+jk2\pi)}dt \\ \\
\ = \frac{-1}{1+jk2\pi} \left[e^{-1}\underbrace{e^{-jk2\pi}}_{cos(2k\pi)}-1\right] = \frac{1}{1+jk2\pi} \left(1-e^{-1}\right) \\ \\
\Rightarrow x(t) = \displaystyle\sum_{k=-\infty}^{\infty} \frac{1}{1+jk2\pi} \left(1-e^{-1}\right)  e^{jk2\pi t} \end{math}  
    \subsection*{e)}
    \begin{math}
    \textrm{Recall that:} \quad \boxed{y(t) = h(t)*x(t)} \quad \boxed{x(t)*\delta(t) = x(t)} \quad \boxed{x(t)*\delta(t-t_0) = x(t-t_0)} \\
    \Rightarrow y(t) = x(t)*(\delta(t) - \delta(t-\frac{1}{2})) \stackrel{\text{(by distribution property)}}{=} x(t)*\delta(t) - x(t)*\delta(t-\frac{1}{2}) \\  
    \Rightarrow y(t) = x(t) - x(t-\frac{1}{2}) \\ 
    = \displaystyle\sum_{k=-\infty}^{\infty} \frac{cos(k\pi)}{1+jk2\pi} \left(e^{\frac{1}{2}}-e^{\frac{-1}{2}}\right)  e^{jk2\pi t} - \displaystyle\sum_{k=-\infty}^{\infty} \frac{cos(k\pi)}{1+jk2\pi} \left(e^{\frac{1}{2}}-e^{\frac{-1}{2}}\right)  e^{jk2\pi (t-\frac{1}{2})} \\
    = \displaystyle\sum_{k=-\infty}^{\infty} \frac{cos(k\pi)}{1+jk2\pi} \left(e^{\frac{1}{2}}-e^{\frac{-1}{2}}\right)  e^{jk2\pi t} - \displaystyle\sum_{k=-\infty}^{\infty} \frac{cos(k\pi)}{1+jk2\pi} \left(e^{\frac{1}{2}}-e^{\frac{-1}{2}}\right)  e^{jk2\pi t} \underbrace{e^{-jk\pi}}_{cos(k\pi)} \\
= \displaystyle\sum_{k=-\infty}^{\infty} \frac{cos(k\pi)}{1+jk2\pi} \left(e^{\frac{1}{2}}-e^{\frac{-1}{2}}\right)  e^{jk2\pi t} \left(1-cos(k\pi)\right) \\
= \displaystyle\sum_{k=-\infty}^{\infty} \frac{cos(k\pi) - \overbrace{cos(k\pi)^{2}}^{1}}{1+jk2\pi} \left(e^{\frac{1}{2}}-e^{\frac{-1}{2}}\right) e^{jk2\pi t} \\
y(t) = \displaystyle\sum_{k=-\infty}^{\infty} \frac{cos(k\pi)-1}{1+jk2\pi} \left(e^{\frac{1}{2}}-e^{\frac{-1}{2}}\right)  e^{jk2\pi t}
\end{math}
    \section*{Question 4}
    \subsection*{a)}
    Note that, any arbitrary periodic signal x(t) (or, x[n]) with Fourier series coefficients \(a_k\) is real-valued if  \(a^{*}_k = a_{-k}\). \\ \\
    For \(x_1(t)\): \ 
    \begin{math}
     a^{*}_k =  [(\frac{1}{2})^{-k}]^{*} = (\frac{1}{2})^{-k} \neq (\frac{1}{2})^{-(-k)}  = (\frac{1}{2})^{k} = a_{-k} \end{math} \\
    For \(x_2(t)\): \ 
    \begin{math}
     a^{*}_k =  (cos(k\pi))^{*} = cos(k\pi) = cos(-k\pi) = a_{-k} \end{math} \\
    For \(x_3[n]\): \ 
    \begin{math}
     a^{*}_k =  (jsin(\frac{k\pi}{2}))^{*} = -jsin(\frac{k\pi}{2})= jsin(\frac{-k\pi}{2}) = a_{-k} \end{math} \\ \\
    Thus, \(x_2(t)\) and \(x_3[n]\) are real-valued signals.
      
    \subsection*{b)}
    Any arbitrary periodic signal x(t) (or, x[n]) with Fourier series coefficients \(a_k\) is even if \(a_k\) is a real-valued and even function.\\
    For \(x_1(t)\): \ 
    \begin{math}
     a_k =  (\frac{1}{2})^{-k}  \Rightarrow \end{math} is a real-valued but not even function, i.e., \begin{math} (\frac{1}{2})^{-k} \Big|_{k=1} \neq (\frac{1}{2})^{-k} \Big|_{k=-1} \end{math}\\
    For \(x_2(t)\): \ 
    \begin{math}
     a_k =  cos(k\pi) \Rightarrow \end{math} is a real-valued and also an even function since consider that \begin{math} -1\leq cos(k\pi) \leq 1 \end{math} \ and \begin{math}\ cos(k\pi)\Big|_{k=1} = cos(k\pi)\Big|_{k=-1}\end{math} \\
    For \(x_3[n]\): \ 
    \begin{math}
     a_k =  jsin(\frac{k\pi}{2}) \Rightarrow  \end{math} neither real-valued nor even function. \\ \\
    Therefore, only the signal \(x_2(t)\) is even. 
      
    \subsection*{c)}
    By the time-shifting property of Continuous Time Fourier Series: \\
    \begin{math} x_2(t-5) = \displaystyle\sum_{k=-100}^{100} cos(k\pi) e^{-jk\frac{\pi}{5}}e^{jk\frac{2\pi}{50}t}  \quad \textrm{and, also note that} \quad cos(k\pi) = \frac{1}{2}(e^{jk\pi}+e^{-jk\pi}) \\
    \Rightarrow x_2(t-5) = \displaystyle\sum_{k=-100}^{100} \frac{1}{2}(e^{jk\frac{4\pi}{5}}+e^{-jk\frac{6\pi}{5}}) e^{jk\frac{2\pi}{50}t} \end{math} \\
    Thus, \(x_2(t-5)\) is another periodic signal with Fourier series coefficients \(b_k = \frac{1}{2}(e^{jk\frac{4\pi}{5}}+e^{-jk\frac{6\pi}{5}})\). \\   
    Also, remember the theorem mentioned above, the signal \(x_2(t-5)\) is real if  \(b^{*}_k = b_{-k}\). \\   
    \begin{math} b^{*}_k = (\frac{1}{2}(e^{jk\frac{4\pi}{5}}+e^{-jk\frac{6\pi}{5}}))^{*} = \frac{1}{2}(e^{-jk\frac{4\pi}{5}}+e^{jk\frac{6\pi}{5}}) = b_{-k} \Rightarrow \end{math} the signal \(x_2(t-5)\) is real. \\ \\
To check whether the signal \(x_2(t-5)\) is even, \(b_k\) can be written as: \\
\begin{math} b_k = \frac{1}{2}[cos(k\frac{4\pi}{5}) + jsin(k\frac{4\pi}{5})+ cos(k\frac{6\pi}{5}) - jsin(k\frac{6\pi}{5})] \\
b_k = \frac{1}{2} [cos(k\frac{4\pi}{5})+cos(k\frac{6\pi}{5})] - \frac{1}{2}j[sin(k\frac{6\pi}{5})-sin(k\frac{4\pi}{5})] \end{math} \\ \\
Also, note that \begin{math} \boxed{sina-sinb = 2cos(\frac{a+b}{2})sin(\frac{a-b}{2})} \\ \\ 
\textrm{Thus,} \ b_k = \frac{1}{2}[cos(k\frac{4\pi}{5})+cos(k\frac{6\pi}{5})] - j[cos(2\pi k)sin(k\frac{2\pi}{5})] \\
\Rightarrow b_k = \frac{1}{2}(cos(k\frac{4\pi}{5})+cos(k\frac{6\pi}{5})) - jsin(k\frac{2\pi}{5}) \end{math} \\ 
As a result, it can be seen that \(b_k\) is not real-valued, so the signal \(x_2(t-5)\) is not even. 
 \subsection*{d)}
By differentiation property of Continuous Time Fourier Series: \\
\begin{math} \frac{d}{dt}x_2(t) = \displaystyle\sum_{k=-\infty}^{\infty} cos(k\pi) jk\frac{2\pi}{50} e^{jk\frac{2\pi}{50}} \Rightarrow \textrm{is a periodic signal with Fourier series coefficients}\ c_k \\ 
\textrm{such that}\ c_k = cos(k\pi) jk\frac{2\pi}{50}. \\
c^{*}_k = -jk\frac{2\pi}{50}cos(k\pi) \quad \textrm{and} \quad c_{-k} = j(-k)\frac{2\pi}{50}cos(-k\pi) = -jk\frac{2\pi}{50}cos(k\pi) \\ 
c^{*}_k = c_{-k} \Rightarrow \frac{d}{dt}x_2(t) \textrm{is real-valued.} \end{math} \\ \\
Note that, \(cos(k\pi)\) take only the values \(+1\) or \(-1\), so it is always real-valued. However, \(-jk\frac{2\pi}{50}\) is purely imaginary. Therefore, \(\frac{d}{dt}x_2(t)\) is not even.
\subsection*{e)} 
\textbf{Parseval's Identity:}
For any continuous-time periodic signal \(x(t)\) with the Fourier series coefficients \(a_k\), \\ 
\begin{math} \frac{1}{T_0} \displaystyle\int_{t_0}^{t_0+T_0}x(t)x^{*}(t)dt = \displaystyle\sum_{k=-\infty}^{\infty}a_k a^{*}_k \Rightarrow \frac{1}{T_0} \displaystyle\int_{t_0}^{t_0+T_0}|x(t)|^{2} dt = \displaystyle\sum_{k=-\infty}^{\infty}|a_k |^{2} \end{math} \\ \\
In this question, \(x_1(t)\) is a periodic signal whose fundamental period, \(T_0 = 50\) and its Fourier series coefficients \(a_k = (\frac{1}{2})^{-k}\). Please also note that, for this signal, the coefficients are defined only for \(0\leq k\leq100\) since outside this domain all of the coefficients are equal to 0. So, by using Parseval's identity, the average power of \(x_1(t)\) in one period: \\ 
\begin{math} \frac{1}{50} \displaystyle\int_{0}^{50}|x_1(t)|^{2}dt = \displaystyle\sum_{k=0}^{100}|a_k|^{2} = \displaystyle\sum_{k=0}^{100}\left(\frac{1}{2}\right)^{-2k} = \displaystyle\sum_{k=0}^{100}4^{k} = 1+4^{1}+4^{2}+4^{3}+..........+4^{100} \end{math} 
 
 

\end{document}

