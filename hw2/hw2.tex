\documentclass[12pt]{article}
\usepackage[utf8]{inputenc, }
\usepackage{graphicx}
\usepackage{svg}
\usepackage{hyperref}
\usepackage[margin=1in]{geometry}
\usepackage{setspace}
\usepackage{color}
\usepackage{pdfpages}
\usepackage{amsmath}
\usepackage{float}
\usepackage{tikz}
\usepackage{pgfplots}
% \usepackage[nocheck]{fancyhdr}

\sloppy
\definecolor{lightgray}{gray}{0.5}
\setlength{\parindent}{0pt}

\hypersetup{
    colorlinks,
    citecolor=black,
    filecolor=black,
    linkcolor=black,
    urlcolor=black
    pdftitle={EE300 İsmail Enes Bülbül}
}
\onehalfspacing

% \raggedright

\title{EE301 Homework-2}
\author{İsmail Enes Bülbül, Eren Meydanlı, Ahmet Caner Akar}
%\date{October 2022}
\renewcommand*\contentsname{Table of Contents}
\renewcommand*{\refname}{}
% \fancyhf{} % sets both header and footer to nothing
% \renewcommand{\headrulewidth}{0pt}
\begin{document}

\maketitle
% \tableofcontents
% \newpage


    \section*{Question 1}
    \subsection*{a)}
    Let \(x[n] = \delta[n]\). Then \(y[n] = h[n]\)\\
    \begin{math}
      h[n] - ah[n-1] = \delta[n] - b\delta[n-1]
    \end{math}\\
    We know that \(h[n] = 0\) for \(n < 0\) as the system is causal.\\
    For \(n = 0\) we have\\
    \begin{math}
      h[0] - ah[-1] = \delta[0] - b\delta[-1]\\
      h[0] = 1
    \end{math}\\
    For \(n = 1\), we have\\
    \begin{math}
      h[1] - ah[0] = \delta[1] - b\delta[0]\\
      h[1] - a = -b\\
      h[1] = a - b
    \end{math}\\
    For \(n = 2\), we have\\
    \begin{math}
      h[2] - ah[1] = \delta[2] - b\delta[1]\\
      h[2] - a(a-b) = 0\\
      h[2] = a(a-b)
    \end{math}\\
    For \(n = 3\), we have\\
    \begin{math}
      h[3] - ah[2] = \delta[3] - b\delta[2]\\
      h[3] - a^2(a-b) = 0\\
      h[2] = a^2(a-b)
    \end{math}\\
    In general, for \(n > 0\), we have \(h[n] = a^{n-1}(a-b)\).\\
    \begin{math}
      h[n] = \begin{cases}
        a^{n-1}(a-b),& n>0 \\
        1,& n=0\\
        0,& n<0\\
      \end{cases}
    \end{math}\\
    If the system is stable, then\\
    \begin{math}
      \sum_{k = -\infty}^{\infty}|h[k]| < \infty\\
      1 + \sum_{k = 1}^{\infty}|a|^{k-1}|a-b| < \infty  \\
      1 + |a-b|\sum_{k = 0}^{\infty}|a|^k < \infty\\
    \end{math}
    Therefore, \(-1< a < 1\) if the system is stable.
    \subsection*{b)}
    
    
    \section*{Question 2}
    \subsection*{a)}
	\(x(t)\) and \(h(t)\) can be written as: \\
    \begin{math}
      x(t) = u(t+1)-u(t-1) \\
      h(t) =  (1-t)[u(t)-u(t-1)] = \begin{cases}
      1-t,& 0<t<1 \\
      0,& otherwise
    \end{cases} \end{math}\\
      Then, \begin{math}y(t) = x(t)*h(t) \end{math}\\ \\
      Before evaluating the convolution, consider the following: \\
     \begin{math} \hat{y}(t) = u(t)*h(t) = \int_{-\infty}^{\infty} u(t-\tau)h(\tau) \,d\tau \end{math} \\
     \begin{math}  = \int_{-\infty}^{t} h(\tau) \,d\tau = \begin{cases}
      0,& t < 0\\
      \int_{0}^{t} (1-\tau) \,d\tau,& 0 < t < 1\\
     \int_{0}^{1} (1-\tau) \,d\tau,& t > 1
    \end{cases} \ =   \begin{cases}
      0,& t < 0\\
      t - \frac{t^{2}}{2} ,& 0 < t < 1\\
     \frac{1}{2},& t > 1
    \end{cases}  \end{math} \\
    By the properties of the LTI system: \\ 
    \begin{math}
      \resizebox{.9\hsize}{!}{$
      \hat{y}(t+1) = u(t+1)*h(t) = \begin{cases}
      0,& t+1 < 0\\
      (t+1) - \frac{(t+1)^{2}}{2},& 0 < t+1 <1\\
     \frac{1}{2},& t+1 > 1
    \end{cases} \ =  \begin{cases}
      0,& t < -1\\
      (t+1) - \frac{(t+1)^{2}}{2},& -1 < t < 0\\
     \frac{1}{2},& t > 0
    \end{cases}$}  \\ \\ \\
    \resizebox{.9\hsize}{!}{$
    \hat{y}(t-1) = u(t-1)*h(t) = \begin{cases}
      0,& t-1 < 0\\
      (t-1) - \frac{(t-1)^{2}}{2},& 0 < t-1 <1\\
     \frac{1}{2},& t-1 > 1
    \end{cases} \ =  \begin{cases}
      0,& t < 1\\
      (t-1) - \frac{(t-1)^{2}}{2},& 1 < t < 2\\
     \frac{1}{2},& t > 2
    \end{cases}$} \end{math}\\ \\ \\
    Therefore, \begin{math} y(t) = \hat{y}(t+1) - \hat{y}(t-1) =  \begin{cases}
      0,& t < -1\\
      (t+1) - \frac{(t+1)^{2}}{2},& -1 < t < 0\\
     \frac{1}{2},& 0 < t < 1\\
     \frac{1}{2} - [(t-1) - \frac{(t-1)^{2}}{2}],& 1 < t < 2\\
     0,& t > 2
    \end{cases}  \end{math} \\

    \subsection*{b)} \begin{math}
    w(t) = h(t)*g(t) \\ 
    g(t)\end{math} can be written as: \begin{math} g(t) = x(t) + x(t-1) - x(t+1) \end{math} \\
    Then, \begin{math} w(t) = h(t)*[x(t) + x(t-1) - x(t+1)] \end{math} \\ \\
    By the distributive property of the convolution over addition: \\
    \begin{math} w(t) = h(t)*x(t) + h(t)*x(t-1) - h(t)*x(t+1)\\
    = y(t) + y(t-1) - y(t+1) \end{math} [by considering part a] \\
   Acik formulu yazilacak \\  
				   	 
    \section*{Question 3}
    \subsection*{a)}
    Recall that in continuous-time systems one can apply the derivative
    operation to any arbitrary input signal. For example,\\
    \begin{math}
      \delta(t) = \frac{d}{dt}u(t)\longrightarrow \int_{-\infty }^{\infty }  \delta(t)\,dt 
    \end{math}\\
    Also, it can be applied by using the formal definition of the derivative:\\
    \begin{math}
      \lim_{h \to 0} \frac{x(t+h)-x(t)}{h} 
    \end{math}\\
    However, in discrete-time systems h cannot go to zero and the minimum
    value for h can be one. So, in discrete time the derivative expression becomes\\
    \begin{math}
      \lim_{h \to 1} \frac{x(n+h)-x(n)}{h} = x[n+1]-x[n] 
    \end{math}\\
    Thus, the derivative operation in continuous-time systems is analogous of
    the difference operation in discrete-time. Hence, we can obtain the impulse
    response of the difference operation as: \(h[n] = \delta[n] - \delta[n-1]\)
    \subsection*{b)}
    By convolution, \(y[n] = x[n]*h[n] = x[n]*(\delta[n]-\delta[n-1])\)\\
    By distributive property of the convolution operation,\\ 
    \(x[n]*( \delta[n] – \delta[n-1]) = (x[n]* \delta[n]) – (x[n]* \delta[n-1])\)\\
    \(y[n] = x[n] – x[n-1]\)
    \subsection*{c)}
    \begin{math}
      e^{j\Omega_0n}(1-e^{-j\Omega_0}) = (cos(\Omega_0n) +jsin(\Omega_0n))(1-e^{-j\Omega_0}) \\
      =cos(\Omega_0n) + jsin(\Omega_0n) - (cos(\Omega_0n) + jsin(\Omega_0n))(cos(\Omega_0n) - jsin(\Omega_0n)) \\
      =cos(\Omega_0n) + jsin(\Omega_0n) - cos(\Omega_0n)cos(\Omega_0) + jcos(\Omega_0n)sin(\Omega_0) - jsin(\Omega_0n)cos(\Omega_0) - sin(\Omega_0n)sin(\Omega_0) \\
      =cos(\Omega_0n) + jsin(\Omega_0n) - (cos(\Omega_0n)cos(\Omega_0) + sin(\Omega_0n)sin(\Omega_0)) -j(sin(\Omega_0n)cos(\Omega_0) - cos(\Omega_0n)sin(\Omega_0)) \\
      =cos(\Omega_0n) + jsin(\Omega_0n) - cos(\Omega_0n - \Omega_0) -jsin(\Omega_0n - \Omega_0) \\
      =cos(\Omega_0n) - cos(\Omega_0(n-1)) + j(sin(\Omega_0n) - sin(\Omega_0(n-1))) \\ \\
      \left\lvert y[n]\right\rvert = [(cos(\Omega_0n) - cos(\Omega_0(n-1)))^2+(sin(\Omega_0n) - sin(\Omega_0(n-1)))^2]^{1/2} \\ 
      = [cos^2(\Omega_0n) - 2cos(\Omega_0n)cos(\Omega_o(n-1)) + cos^2(\Omega_0(n-1))  +  sin^2(\Omega_0n) - 2sin(\Omega_0n)sin(\Omega_o(n-1)) + sin^2(\Omega_0(n-1))]^{1/2} \\
      = \sqrt{2 - 2[cos(\Omega_0n)cos(\Omega_0(n-1)) + sin(\Omega_0n)sin(\Omega_0(n-1))]} \\
      = \sqrt{2-2cos(\Omega_0n-\Omega_0(n-1))} \\
      = \sqrt{2-2cos(\Omega_0)} \\       
    \end{math}
    \subsection*{d)}
    \begin{math}
      \Omega_0 = 0 \longrightarrow \left\lvert y[n]\right\rvert = \sqrt{2-2cos(0)} = 0 \\
      \Omega_0 = \frac{2\pi}{8} \longrightarrow \left\lvert y[n]\right\rvert = \sqrt{2-2cos(\frac{2\pi}{8})} = \sqrt{2-\sqrt{2}} \\
      \Omega_0 = \frac{2\pi}{4} \longrightarrow \left\lvert y[n]\right\rvert = \sqrt{2-2cos(\frac{2\pi}{4})} = \sqrt{2} \\
      \Omega_0 = \frac{2\pi}{2} \longrightarrow \left\lvert y[n]\right\rvert = \sqrt{2-2cos(\frac{2\pi}{2})} = 2 \\
    \end{math} \\
As we have shown in part a), the given discrete-time LTI system is analagous to the derivative operation in continuous-time systems. Also, please remember that the derivative is defined as the rate of change of a function with respect to a variable. Here, as the frequency of the input x[n] is increased, the period becomes smaller and the values of the input function are changed more rapidly. Therefore, the rate of change of a input function is increased as the frequency is increased, and the modulus of the output \(|y[n]|\) is getting larger values.
    \subsection*{e)}
    \(x[n] = u[n+5] - u[n-2]\) can be plotted as:\\
    \begin{tikzpicture}
        \begin{axis}[
            axis lines=middle,
            xlabel={$n$},
            xlabel style={xshift=0.4cm},
            ylabel={$x[n]$},
            ylabel style={yshift=0.4cm},
	    xmin=-7, xmax=4.5,
            ymin=-1, ymax=1.2,
            xticklabel style={
              anchor=north,
              inner sep=1pt,
            }
          ]
         % \addplot [ycomb, black, thick, mark=*] table [x={n}, y={xn}] {data3ex.dat};
        \end{axis}
      \end{tikzpicture}\\
    We can calculate \(y[n]\) for \(x[n] = u[n+5] - u[n-2]\) by using the answer in 
    (b) as:\\
    \begin{math}
      y[n] = u[n+5] - u[n+2] - (u[n+4] + u[n-3]) 
    \end{math}\\ 
    \(y[n]\) can be plotted as:\\
    \begin{tikzpicture}
      \begin{axis}[
          axis lines=middle,
          xlabel={$n$},
          xlabel style={xshift=0.4cm},
	  ylabel={$y[n]$},
          ylabel style={yshift=0.4cm},
          xmin=-7, xmax=4.5,
          ymin=-1, ymax=1.2,
          xticklabel style={
            anchor=north east,
            inner sep=1pt,
          }
        ]
        %\addplot [ycomb, black, thick, mark=*] table [x={n}, y={xn}] {data3ey.dat};
      \end{axis}
    \end{tikzpicture}\\
    From the graph of \(y[n]\), we can say that \(x[n]\) has edges at \(n=-5\) 
    and \(n=2\). 

    \subsection*{f)}
    From (d), we can say that magnitude of the output signal of the system 
    increases as the frequency increases. In (f), \(x[n]\) has rapid changes at 
    \(n=-5\) and \(n=2\), and as a result, \(y[n]\) has magnitude 1 at \(n=-5\) and 
    \(n=2\), and 0 elsewhere.

    \section*{Question 4}
    \subsection*{a)}
    Note that, any arbitrary periodic signal x(t) (or, x[n]) with Fourier series coefficients \(a_k\) is real-valued if  \(a^{*}_k = a_{-k}\). \\ \\
    For \(x_1(t)\): \ 
    \begin{math}
     a^{*}_k =  [(\frac{1}{2})^{-k}]^{*} = (\frac{1}{2})^{-k} \neq (\frac{1}{2})^{-(-k)}  = (\frac{1}{2})^{k} = a_{-k} \end{math} \\
    For \(x_2(t)\): \ 
    \begin{math}
     a^{*}_k =  (cos(k\pi))^{*} = cos(k\pi) = cos(-k\pi) = a_{-k} \end{math} \\
    For \(x_3[n]\): \ 
    \begin{math}
     a^{*}_k =  (jsin(\frac{k\pi}{2}))^{*} = -jsin(\frac{k\pi}{2})= jsin(\frac{-k\pi}{2}) = a_{-k} \end{math} \\ \\
    Thus, \(x_2(t)\) and \(x_3[n]\) are real-valued signals.
      
    \subsection*{b)}
    Any arbitrary periodic signal x(t) (or, x[n]) with Fourier series coefficients \(a_k\) is even if \(a_k\) is a real-valued and even function.\\
    For \(x_1(t)\): \ 
    \begin{math}
     a_k =  (\frac{1}{2})^{-k}  \Rightarrow \end{math} is a real-valued but not even function, i.e., \begin{math} (\frac{1}{2})^{-k} \Big|_{k=1} \neq (\frac{1}{2})^{-k} \Big|_{k=-1} \end{math}\\
    For \(x_2(t)\): \ 
    \begin{math}
     a_k =  cos(k\pi) \Rightarrow \end{math} is a real-valued and also an even function since consider that \begin{math} -1\leq cos(k\pi) \leq 1 \end{math} \ and \begin{math}\ cos(k\pi)\Big|_{k=1} = cos(k\pi)\Big|_{k=-1}\end{math} \\
    For \(x_3[n]\): \ 
    \begin{math}
     a_k =  jsin(\frac{k\pi}{2}) \Rightarrow  \end{math} neither real-valued nor even function. \\ \\
    Therefore, only the signal \(x_2(t)\) is even. 
      
    \subsection*{c)}
    By the time-shifting property of Continuous Time Fourier Series: \\
    \begin{math} x_2(t-5) = \displaystyle\sum_{k=-100}^{100} cos(k\pi) e^{-jk\frac{\pi}{5}}e^{jk\frac{2\pi}{50}t}  \quad \textrm{and, also note that} \quad cos(k\pi) = \frac{1}{2}(e^{jk\pi}+e^{-jk\pi}) \\
    \Rightarrow x_2(t-5) = \displaystyle\sum_{k=-100}^{100} \frac{1}{2}(e^{jk\frac{4\pi}{5}}+e^{-jk\frac{6\pi}{5}}) e^{jk\frac{2\pi}{50}t} \end{math} \\
    Thus, \(x_2(t-5)\) is another periodic signal with Fourier series coefficients \(b_k = \frac{1}{2}(e^{jk\frac{4\pi}{5}}+e^{-jk\frac{6\pi}{5}})\). \\   
    Also, remember the theorem mentioned above, the signal \(x_2(t-5)\) is real if  \(b^{*}_k = b_{-k}\). \\   
    \begin{math} b^{*}_k = (\frac{1}{2}(e^{jk\frac{4\pi}{5}}+e^{-jk\frac{6\pi}{5}}))^{*} = \frac{1}{2}(e^{-jk\frac{4\pi}{5}}+e^{jk\frac{6\pi}{5}}) = b_{-k} \Rightarrow \end{math} the signal \(x_2(t-5)\) is real. \\ \\
To check whether the signal \(x_2(t-5)\) is even, \(b_k\) can be written as: \\
\begin{math} b_k = \frac{1}{2}[cos(k\frac{4\pi}{5}) + jsin(k\frac{4\pi}{5})+ cos(k\frac{6\pi}{5}) - jsin(k\frac{6\pi}{5})] \\
b_k = \frac{1}{2} [cos(k\frac{4\pi}{5})+cos(k\frac{6\pi}{5})] - \frac{1}{2}j[sin(k\frac{6\pi}{5})-sin(k\frac{4\pi}{5})] \end{math} \\ \\
Also, note that \begin{math} \boxed{sina-sinb = 2cos(\frac{a+b}{2})sin(\frac{a-b}{2})} \\ \\ 
\textrm{Thus,} \ b_k = \frac{1}{2}[cos(k\frac{4\pi}{5})+cos(k\frac{6\pi}{5})] - j[cos(2\pi k)sin(k\frac{2\pi}{5})] \\
\Rightarrow b_k = \frac{1}{2}(cos(k\frac{4\pi}{5})+cos(k\frac{6\pi}{5})) - jsin(k\frac{2\pi}{5}) \end{math} \\ 
As a result, it can be seen that \(b_k\) is not real-valued, so the signal \(x_2(t-5)\) is not even. 
 \subsection*{d)}
By differentiation property of Continuous Time Fourier Series: \\
\begin{math} \frac{d}{dt}x_2(t) = \displaystyle\sum_{k=-\infty}^{\infty} cos(k\pi) jk\frac{2\pi}{50} e^{jk\frac{2\pi}{50}} \Rightarrow \textrm{is a periodic signal with Fourier series coefficients}\ c_k \\ 
\textrm{such that}\ c_k = cos(k\pi) jk\frac{2\pi}{50}. \\
c^{*}_k = -jk\frac{2\pi}{50}cos(k\pi) \quad \textrm{and} \quad c_{-k} = j(-k)\frac{2\pi}{50}cos(-k\pi) = -jk\frac{2\pi}{50}cos(k\pi) \\ 
c^{*}_k = c_{-k} \Rightarrow \frac{d}{dt}x_2(t) \textrm{is real-valued.} \end{math} \\ \\
Note that, \(cos(k\pi)\) take only the values \(+1\) or \(-1\), so it is always real-valued. However, \(-jk\frac{2\pi}{50}\) is purely imaginary. Therefore, \(\frac{d}{dt}x_2(t)\) is not even.
\subsection*{e)} 
\textbf{Parseval's Identity:}
For any continuous-time periodic signal \(x(t)\) with the Fourier series coefficients \(a_k\), \\ 
\begin{math} \frac{1}{T_0} \displaystyle\int_{t_0}^{t_0+T_0}x(t)x^{*}(t)dt = \displaystyle\sum_{k=-\infty}^{\infty}a_k a^{*}_k \Rightarrow \frac{1}{T_0} \displaystyle\int_{t_0}^{t_0+T_0}|x(t)|^{2} dt = \displaystyle\sum_{k=-\infty}^{\infty}|a_k |^{2} \end{math} \\ \\
In this question, \(x_1(t)\) is a periodic signal whose fundamental period, \(T_0 = 50\) and its Fourier series coefficients \(a_k = (\frac{1}{2})^{-k}\). Please also note that, for this signal, the coefficients are defined only for \(0\leq k\leq100\) since outside this domain all of the coefficients are equal to 0. So, by using Parseval's identity, the average power of \(x_1(t)\) in one period: \\ 
\begin{math} \frac{1}{50} \displaystyle\int_{0}^{50}|x_1(t)|^{2}dt = \displaystyle\sum_{k=0}^{100}|a_k|^{2} = \displaystyle\sum_{k=0}^{100}\left(\frac{1}{2}\right)^{-2k} = \displaystyle\sum_{k=0}^{100}4^{k} = 1+4^{1}+4^{2}+4^{3}+..........+4^{100} \end{math} 
 
    \section*{Question 5}
    \subsection*{a)}
In calculating part of the question, we added zero vectors to end and beginning of the main vector to obtain correct results in range of \(0\leq n \leq N-1\), which can be seen in Appendix. Since we manually did that zero adding part, we deviated the result for \(n>N\) range. Because of this situation, our calculated result has only N value in range. On the other hand, MATLAB’s built-in function, \emph{conv()}, evaluate the convolution so that the size of the output is equal to \(N+L-1\). That's why the output graphs of convolution operation are different by using our function and MATLAB's built-in function as shown in Appendix. In fact, to get the same result that we got by using our function, one can specify an input argument, \emph{same}, to MATLAB's function so that the result gives the central part of the convolution operation. Otherwise, the built-in function gives the full convolution whose length is equal to \(N+L-1\).
\subsection*{b)}
The output signal looks more smoother in larger values of L. As we can observe that h[n] impulse response created accordingly to L which directly affects the convolution range. By saying range, we mean that how many \(x*h\) multiplication we will use to calculate one element of the output. So, this is why that the output signal gain more elements and more precise values when L gets larger. And also, we can observe that the L value expands the output range by expanding h[n] size. The MATLAB code and the resulting graphs for different values of L can be seen in Appendix. 

    

\end{document}



%% TODO
%
% 1 yaz
% 2 yaz
% 3 yaz
% 5 commentle ve yaz
% örnek appendix kod ekle
% plotları düzenle

%% kontroller
% 1 kontrol et
% ...


%% eksikler (gruba yazılacak)
% 2 integraller yeniden hesaplanmalı
% 2 comment yaz
% 5 kodları iste
